\documentclass[]{article}
\usepackage{lmodern}
\usepackage{amssymb,amsmath}
\usepackage{ifxetex,ifluatex}
\usepackage{fixltx2e} % provides \textsubscript
\ifnum 0\ifxetex 1\fi\ifluatex 1\fi=0 % if pdftex
  \usepackage[T1]{fontenc}
  \usepackage[utf8]{inputenc}
\else % if luatex or xelatex
  \ifxetex
    \usepackage{mathspec}
    \usepackage{xltxtra,xunicode}
  \else
    \usepackage{fontspec}
  \fi
  \defaultfontfeatures{Mapping=tex-text,Scale=MatchLowercase}
  \newcommand{\euro}{€}
    \setmainfont{PT Serif}
\fi
% use upquote if available, for straight quotes in verbatim environments
\IfFileExists{upquote.sty}{\usepackage{upquote}}{}
% use microtype if available
\IfFileExists{microtype.sty}{%
\usepackage{microtype}
\UseMicrotypeSet[protrusion]{basicmath} % disable protrusion for tt fonts
}{}
\ifxetex
  \usepackage[setpagesize=false, % page size defined by xetex
              unicode=false, % unicode breaks when used with xetex
              xetex]{hyperref}
\else
  \usepackage[unicode=true]{hyperref}
\fi
\hypersetup{breaklinks=true,
            bookmarks=true,
            pdfauthor={David Charte},
            pdftitle={Common multilabel formulae},
            colorlinks=true,
            citecolor=blue,
            urlcolor=blue,
            linkcolor=magenta,
            pdfborder={0 0 0}}
\urlstyle{same}  % don't use monospace font for urls
\setlength{\parindent}{0pt}
\setlength{\parskip}{6pt plus 2pt minus 1pt}
\setlength{\emergencystretch}{3em}  % prevent overfull lines
\setcounter{secnumdepth}{0}

\title{Common multilabel formulae\\\vspace{0.5em}{\large \ldots{}under a common notation}}
\author{David Charte}
\date{}
\newcommand{\func}[1]{\operatorname{\mathit{#1}}} \newcommand{\const}[1]{\mathit{#1}} \newcommand{\abs}[1]{\left\lvert{#1}\right\rvert}

\begin{document}
\maketitle
\begin{abstract}
Multilabel Classification is a branch of Data Mining in which many
different metrics and equations are defined. Each author uses different
notation and basic definitions, thus difficulting the coherence of
future papers where some of these formulas are used. This is a document
where the multilabel scenario is presented with simple basic
definitions, and a list of commonly used formulas are rigurously adapted
to this notation, in an attempt to introduce them in a clear way.
\end{abstract}

\section{Definitions}\label{definitions}

Let \(A_1, A_2, \dots A_f\) be arbitrary sets. We will call them
\emph{input attributes} or simply \emph{attributes}. An instance will
take a certain value on each of these sets, that is, we will be working
with elements of their cartesian product,
\(A_1\times A_2\times\dots\times A_f\).

Let \(L\) be a finite set. This will be the set of all possible labels.
Each instance of a dataset will then have a subset of active labels or
\emph{labelset}, \(y \subset L\).

Let \(D\) be a finite subset of
\(A_1\times A_2\times\dots\times A_f\times\mathcal{P}(L)\), where
\(\mathcal{P}(L)\) is the powerset of \(L\), that is, the set of all
possible combinations of labels. We will call \(D\) a \emph{dataset} and
each of its elements an \emph{instance}: \((x, y)\in D\) where
\(x=(x_1,x_2,\dots x_f)\in A_1\times A_2\times\dots\times A_f\) and
\(y=\{l_1,\dots l_k\} \in\mathcal{P}(L)\).

\textbf{Note}: Since \(D\) is a set, we will be assuming no two
instances are identical.

\section{Basic measures}\label{basic-measures}

In the following we will assume \(D\) is a fixed set. Otherwise, many of
the measures would be dependent on D. Some basic data directly extracted
from the definitions are:

\begin{itemize}
\itemsep1pt\parskip0pt\parsep0pt
\item
  Number of input attributes, \(f\)
\item
  Number of labels, \(\abs L\)
\item
  Number of instances, \(\abs D\)
\item
  Number of distinct labelsets, \(\abs{\{Y:(X, Y)\in D\}}\)
\end{itemize}

Many of the measures specific to multilabel classification are related
to the labels themselves. For instance, we can find out the mean number
of labels in a labelset (\emph{Card}, \ref{eq:Card}) and its proportion
as to the total number of labels (\emph{Dens}, \ref{eq:Dens}):

\begin{equation} \func{Card} = \frac 1 {\abs D} \sum_{(x,y)\in D} \abs y \label{eq:Card}\end{equation}

\begin{equation} \func{Dens} = \frac 1 {\abs D \abs L} \sum_{(x,y)\in D} \abs y \label{eq:Dens}\end{equation}

\section{Imbalance and concurrence}\label{imbalance-and-concurrence}

\end{document}
